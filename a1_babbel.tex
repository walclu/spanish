\chapter{Babbel A1}
\section{Ser y estar}
\textbf{Datum: 25.12.2023}
\subsection*{Verwendung von ser}
\textbf{ser} bezieht sich auf \underline{dauerhafte} Eigenschaften - dazu 
gehören physische Eigenschaften von Personen oder Objekten oder
persönliche Eigenschaften.

\begin{ejemplos}
    \item Los carros son grandes.
    \item Ella es alta y amable.
\end{ejemplos}

Ser wird auch benutzt um Nationalitäten und Berufe anzugeben.

\begin{ejemplos}
    \item Nosotros somos italianos.
    \item Ellos son profesores.
\end{ejemplos}

\subsection*{Verwendung von estar}
\textbf{Estar} bezieht sich auf \underline{vorübergehende} Zustände,
Gefühle, Familienstand, Gesundheit oder auch Ortsangaben.

\begin{ejemplos}
    \item Estoy muy feliz.
    \item Tú estás casado con Ana.
    \item Usted está enferma.
    \item Sevilla está en España.
\end{ejemplos}

\section{Presente continuo - Gerundio}
\textbf{Datum: 26.12.2023}\\
Das presente continuo wird \textbf{ausschließlich}
dazu verwendet, um über eine Handlung zu sprechen,
die jetzt gerade stattfindet. Das presente simple
kann hingegen für regelmäßige und gerade stattfindende
Handlungen verwendet werden.

\begin{gramatica}
    \item Bildung: estar + gerundio
\end{gramatica}

\subsection*{Bildung der regelmäßigen Formen}
Um das Gerundium von regelmäßigen Verben zu bilden,
ersetzt man die Endungen \textbf{-ar} durch \textbf{-ando}.
Die Endungen \textbf{-er, ir} werden durch \textbf{-iendo} 
ersetzt.

\begin{ejemplos}
    \item Estoy hablando.
    \item Estamos comiendo.
\end{ejemplos}

\subsection*{Bildung der unregelmäßigen Formen}
Manche Verben haben ein anders gebildetes Gerundium.
Hier einige Beispiele.
\begin{ejemplos}
    \item leer - leyendo
    \item ir - yendo
    \item dormir - durmiendo
    \item oír - oyendo
\end{ejemplos}

\section{Vokalwechsel bei unregelmäßigen Verben}
\textbf{Datum: 27.12.2023}\\
Bei manchen unregelmäßigen Verben findet bei den Formen der Einzahl und der dritten
Person Plural in der Gegenwartsform ein Vokalwechsel statt.
\begin{ejemplos}
    \item v\textbf{o}lar - yo v\textbf{ue}lo
    \item d\textbf{e}cir - ellas d\textbf{i}cen
\end{ejemplos}
Die Formen von nosotros/as und vosotros/as werden jedoch immer \textbf{ohne}
diesen Vokalwechsel gebildet.
\subsection*{Vokalwechsel von o zu ue}
Manche Verben mit einem \textbf{o} in der Grundform wechseln den Vokal zu
\textbf{ue}.
\begin{ejemplos}
    \item t\'u duermes, ella duerma, ellos duermen
    \item dormir, morir, volar, encontrar, poder
\end{ejemplos}
\subsection*{Vokalwechsel von e zu ie}
Einen weiteren Vokalwechsel findet man bei einigen Verben wie sentir. Hier
wird das erste \textbf{e} in der gebeugten Form zu \textbf{ie}.
\begin{ejemplos}
    \item sentir, pensar, entender, perder, mentir, cerrar, pedir, seguir, 
    repetir, sonre\'ir, competir, conseguir
\end{ejemplos}
Hat eins dieser Verben zwei \textbf{e} so wird nur das Zweite geändert.
\section{Llevar + Gerundio}
Die deutsche Übersetzung einer Konstruktion vom Typ llevar + gerundio
müsste im Deutschen in der Regel mit einem Satz übersetzt werden,
der die Präposition \textbf{seit} enthält.
\begin{ejemplos}
    \item Ya llevo dos horas aprendiendo español.
\end{ejemplos}
\section{Pret\'erito perfecto}
\textbf{Datum: 29.12.2023}\\
Das pret\'erito perfecto ist eine Zeitform, mit der man über
die Vergangenheit sprechen kann. Es wird dann verwendet, wenn eine
Handlung in der Vergangenheit begonnen hat und \textbf{bis zur Gegenwart 
andauert.}
\begin{ejemplos}
    \item Este mes \'el ha estado de vacaciones.
\end{ejemplos}
Deshalb wird diese Zeitform meist bei Zeitangaben wie \textbf{hoy},
\textbf{esta semana} oder \textbf{todav\'ia} verwendet.
\begin{ejemplos}
    \item Hoy yo he hablado con mi hermano.
    \item Ellas no han visto esa pel\'icula todav\'ia.
\end{ejemplos}
Man kann diese Zeitform aber auch verwenden, um über eine 
vergangene Erfahrung zu sprechen, die \textbf{für die Gegenwart
relevant ist}.
\begin{ejemplos}
    \item Has estado alguna vez en Estados Unidos?
    \item S\'i, he estado dos veces.
\end{ejemplos}
\begin{gramatica}
    \item haber + participio (Vergangenheit)
\end{gramatica}
\subsection*{Bildung des Partizips für regelmäßige Verben}
\begin{gramatica}
    \item Verben auf \textbf{-ar} → \textbf{-ado}
    \item Verben auf \textbf{-er,-ir} → \textbf{-ido}
\end{gramatica}
\subsection*{Bildung des Partizips für unregelmäßige Verben}
\begin{ejemplos}
    \item hacer - hecho
    \item ir -ido
    \item volver - vuelto
    \item ver - visto
    \item decir - dicho
\end{ejemplos}
\section{Häufigkeitsadverbien}
Häufigkeitsadverbien werden verwendet, um auszudrücken,
wie oft etwas getan wird oder etwas passiert.
\begin{ejemplos}
    \item immer - siempre
    \item normalerweise - normalmente
    \item oft - a menudo
    \item manchmal - a veces
    \item nie - nunca
    \item Fast nie - Casi nunca
    \item einmal pro Monat - una vez al mes
    \item zweimal pro Woche -  dos veces a la semana
\end{ejemplos}
In Aussagesätzen und Fragen steht das Häufigkeitsadverb 
normalerweise entweder \textbf{direkt vor dem Verb} oder 
\textbf{am Ende des Satzes}.
\begin{ejemplos}
    \item Normalmente vas al cine? - Vas al cine normalmente?
    \item Todos los veranos vamos a la playa - Vamos a la playa
    todos los veranos.
\end{ejemplos}
\subsection*{Häufigkeitsadverbien in verneinten Sätzen}
in \textbf{verneinten} Sätzen stehen Häufigkeitsadverbien
fast immer am \underline{Ende des Satzes}.
\begin{ejemplos}
    \item No juegan al f\'utbol todos los dias.
    \item No voy al cine normalmente.
\end{ejemplos}
Stehen sie allerdings am Anfang des verneinten Satzes, so gilt:
\begin{gramatica}
    \item Häufigkeitsadverb + no + Verb
\end{gramatica}

\begin{ejemplos}
    \item Normalmente no voy al cine.
    \item Todos los d\'ias no juegan al f\'utbol. 
\end{ejemplos}
\section{Demonstrativbegleiter}
Demonstrativbegleiter werden verwendet, um den \textbf{Abstand}
zwischen einem Gegenstand und der entsprechenden Person zu 
beschreiben.
\begin{gramatica}
    \item der hier - este
    \item die hier - esta
    \item die hier - estos
    \item die hier - estas
\end{gramatica}
Este und seine Formen verwendest du für Objekte, die sich 
\textbf{nah bei} der sprechenden Person befinden.
\begin{ejemplos}
    \item Este caf\'e de aqui no me gusta.
    \item Esta casa es roja.
\end{ejemplos}
\textit{Este} wird oft zusammen mit \textit{de aqui} (hier) 
verwendet.
\subsection*{Weitere Demonstrativbegleiter}
\textit{Ese} und seine Formen werden verwendet, um sich auf
Personen und Dinge zu beziehen, die räumlich oder zeitlich
\textbf{etwas weiter entfernt} sind.
\begin{ejemplos}
    \item Ese hombre es mi padre.
    \item Esa casa es azul.
    \item Esos libros de ah\'i son de Juan.
\end{ejemplos}
\subsection*{Die Demonstrativbegleiter mit aquel}
\textit{Aquel} verwendet man, um auf Personen oder Dinge 
hinzuweisen, die räumlich oder zeitlich \textbf{weit entfernt}
sind.
\begin{ejemplos}
    \item Aquel coche de all\'i es muy grande.
    \item Aquellas chicas son mis hermanas.
\end{ejemplos}
Diese Demonstrativbegleiter werden oft zusammen mit
\textit{de all\'i} (dort) verwendet. Wird von einem
Ereignis gesprochen, das \textbf{lange} zurückliegt,
werden auch \textit{aquel} und seine Formen verwendet.
