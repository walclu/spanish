\chapter{Babbel A1}
\section{Ser y estar}
\textbf{Datum: 25.12.2023}
\subsection*{Verwendung von ser}
\textbf{ser} bezieht sich auf \underline{dauerhafte} Eigenschaften - dazu 
gehören physische Eigenschaften von Personen oder Objekten oder
persönliche Eigenschaften.

\begin{ejemplos}
    \item Los carros son grandes.
    \item Ella es alta y amable.
\end{ejemplos}

Ser wird auch benutzt um Nationalitäten und Berufe anzugeben.

\begin{ejemplos}
    \item Nosotros somos italianos.
    \item Ellos son profesores.
\end{ejemplos}

\subsection*{Verwendung von estar}
\textbf{Estar} bezieht sich auf \underline{vorübergehende} Zustände,
Gefühle, Familienstand, Gesundheit oder auch Ortsangaben.

\begin{ejemplos}
    \item Estoy muy feliz.
    \item Tú estás casado con Ana.
    \item Usted está enferma.
    \item Sevilla está en España.
\end{ejemplos}

\section{Presente continuo - Gerundio}
\textbf{Datum: 26.12.2023}\\
Das presente continuo wird \textbf{ausschließlich}
dazu verwendet, um über eine Handlung zu sprechen,
die jetzt gerade stattfindet. Das presente simple
kann hingegen für regelmäßige und gerade stattfindende
Handlungen verwendet werden.

\begin{gramatica}
    \item Bildung: estar + gerundio
\end{gramatica}

\subsection*{Bildung der regelmäßigen Formen}
Um das Gerundium von regelmäßigen Verben zu bilden,
ersetzt man die Endungen \textbf{-ar} durch \textbf{-ando}.
Die Endungen \textbf{-er, ir} werden durch \textbf{-iendo} 
ersetzt.

\begin{ejemplos}
    \item Estoy hablando.
    \item Estamos comiendo.
\end{ejemplos}

\subsection*{Bildung der unregelmäßigen Formen}
Manche Verben haben ein anders gebildetes Gerundium.
Hier einige Beispiele.
\begin{ejemplos}
    \item leer - leyendo
    \item ir - yendo
    \item dormir - durmiendo
    \item oír - oyendo
\end{ejemplos}

\section{Vokalwechsel bei unregelmäßigen Verben}
\textbf{Datum: 27.12.2023}\\
Bei manchen unregelmäßigen Verben findet bei den Formen der Einzahl und der dritten
Person Plural in der Gegenwartsform ein Vokalwechsel statt.
\begin{ejemplos}
    \item v\textbf{o}lar - yo v\textbf{ue}lo
    \item d\textbf{e}cir - ellas d\textbf{i}cen
\end{ejemplos}
Die Formen von nosotros/as und vosotros/as werden jedoch immer \textbf{ohne}
diesen Vokalwechsel gebildet.
\subsection*{Vokalwechsel von o zu ue}
Manche Verben mit einem \textbf{o} in der Grundform wechseln den Vokal zu
\textbf{ue}.
\begin{ejemplos}
    \item t\'u duermes, ella duerma, ellos duermen
    \item dormir, morir, volar, encontrar, poder
\end{ejemplos}
\subsection*{Vokalwechsel von e zu ie}
Einen weiteren Vokalwechsel findet man bei einigen Verben wie sentir. Hier
wird das erste \textbf{e} in der gebeugten Form zu \textbf{ie}.
\begin{ejemplos}
    \item sentir, pensar, entender, perder, mentir, cerrar, pedir, seguir, 
    repetir, sonre\'ir, competir, conseguir
\end{ejemplos}
Hat eins dieser Verben zwei \textbf{e} so wird nur das Zweite geändert.
\section{Llevar + Gerundio}
Die deutsche Übersetzung einer Konstruktion vom Typ llevar + gerundio
müsste im Deutschen in der Regel mit einem Satz übersetzt werden,
der die Präposition \textbf{seit} enthält.
\begin{ejemplos}
    \item Ya llevo dos horas aprendiendo español.
\end{ejemplos}





