\chapter{Babbel A2}
\section{Die indirekten Objektpronomen}
\textbf{31.12.20.23}\\
Nach dem \textbf{indirekten Objektpronomen} fragt man mit 
\textbf{wem oder was?}. DIe Objektpronomen für 
\textit{yo, t\'u, nosotros/as} und \textit{vosotros/as} sind 
gleich wie die direkten Objektpronomen.
\begin{ejemplos}
    \item Ella me da la flor.
    \item Andr\'es te escribe cartas?
\end{ejemplos}
\subsection*{le und les}
Die indirekten Objektpronomen für die 3. Person sind 
\underline{anders} als die direkten Objektpronomen und
richten sich allein nach der \underline{Anzahl}.
\begin{ejemplos}
    \item Le compro una manzana.
    \item Ya les hemos tra\'ido sus cosas.
    \item Nos les has comprado el billete.
\end{ejemplos}
\section{Fragewörter}
Es gibt die Fragewörter:
\begin{gramatica}
    \item was, welcher - qu\'e
    \item wie - c\'o
    \item warum - por qu\'e
    \item wann - cu\'ando
    \item d\'onde - wo
    \item wer - qui\'en
    \item wie viel - cu\'anto
\end{gramatica}
\textit{Qui\'en} und \textit{cu\'anto} müssen ans Nomen 
angeglichen werden. Für \textit{qui\'en} gibt eine
Einzahlform und eine Mehrzahlform.
\begin{ejemplos}
    \item Qui\'en eres?
    \item Qui\'enes son ustedes?
\end{ejemplos}
\textit{Cu\'anto} hingegen wird wie ein Artikel angeglichen.
Die Formen sind \textit{cu\'anto, cu\'anta, cu\'antos, cu\'antas}. 
\begin{ejemplos}
    \item Cu\'antas hermanas tienes?
    \item Cuanta gente esta aqui?
\end{ejemplos}
\section{Nomen mit unregelmäßiger Endung}
Einige Berufsbezeichnungen enden in der männlichen und 
weiblichen Form aus \textit{-a}.
\begin{ejemplos}
    \item la periodista - el periodista
    \item la artista - el artista
    \item la dentista - el dentista
\end{ejemplos}
Einige weitere Nomen auf \textit{-a} sind ebenfalls männlich.
\begin{ejemplos}
    \item el tema
    \item el problema
    \item el mapa
    \item el sofa
    \item el idioma
\end{ejemplos}
Einige Wörter, die aus \textit{-o} enden, sind dagegen 
weiblich.
\begin{ejemplos}
    \item la foto
    \item la moto
    \item la mano
\end{ejemplos}
\subsection*{Weitere Endungen}
Männliche Nomen sind durch die Endungen \textit{-or, \'on, -an}
erkennbar.
\begin{ejemplos}
    \item el coraz\'on - el sill\'on
    \item el escritor - el vendedor
    \item el capit\'an
\end{ejemplos}
Nomen mit Endungen auf \textit{-i\'on, -ad, -ez} sind fast
immer weiblich.
\begin{ejemplos}
    \item la canci\'on
    \item la ciudad
    \item la timidez - la ni\~nes
\end{ejemplos}
\section{Relativsätze mit \textit{que}}
\textbf{01.01.2024}\\
\textit{Que} bezieht sich auf ein Nomen, das näher bestimmt
wird.
\begin{ejemplos}
    \item Tienes la revista que le\'imos ayer?
    \item El amigo que me llam\'o por la ma\~nana viene
    ahora.
    \item Los ni\~nos que juegan ah\'i son mis primos.
    \item La maleta que voy a comprar es peque\~na.
\end{ejemplos}
\section{Desde und desde hace}
Die deutsche Präposition \textbf{seit} hat im Spanischen
zwei Bedeutungen. \textit{Desde hace} bezieht sich auf einen
Zeitraum, der in der Vergangenheit angefangen hat.
\begin{ejemplos}
    \item Desde hace dos semanas estoy mejor.
    \item No voy al m\'edico desde hace cuatro mes.
\end{ejemplos}
Ein bestimmter \underline{Zeitpunkt} in der Vergangenheit
wird dagegen mit \textit{desde} angegeben.
\begin{ejemplos}
    \item No fumo desde el a\~no pasado.
    \item Desde el sabado estoy enfermo.
\end{ejemplos}
\subsection*{Zeitangaben mit \textit{hace}}
\textit{Hace} wird wie das Deutsche \textbf{vor} verwendet,
um abgeschlossene Zeiträume in der Vergangenheit zu beschreiben.
\begin{ejemplos}
    \item Hace dos semanas fui al gimnasio.
    \item Comimos hace tres horas.
    \item Hace unos meses estuvo enfermo.
\end{ejemplos}
\section{Imperativ}
\subsection*{Regelmäßige Verben auf -ar}
Die Befehlsform wird verwendet, um jemanden um etwas direkt
zu bitten oder ihn aufzufordern. Wenn man eine Person duzt,
benutzt man einfach die dritte Person Einzahl des Verbes.
\begin{ejemplos}
    \item Jordi no canta bien, canta t\'u.
\end{ejemplos}
Wenn man mehrere Personen duzt und anspricht, wird die 
Endung der Grundform durch \textit{-ad} ersetzt.
\begin{ejemplos}
    \item No pod\'eis jugar aqui, jugad en el parque.
\end{ejemplos}
In Lateinamerika wird hingegen für diese Form 
\textit{usedes} genutzt, also wird die Endung 
der Grundform durch \textit{-en} ersetzt.
\begin{ejemplos}
    \item No pueden cantar aqui, canten en el parque.
\end{ejemplos}
Wenn man mehrere Personen anspricht und sich selbst einbezieht,
wird die Endung der Grundform durch \textit{-emos} ersetzt.
\begin{ejemplos}
    \item Ellos no saben c\'omo limpiar, limpiemos la casa.
\end{ejemplos}
\subsection*{Regelmäßige Verben auf -er}
Wenn man eine Person duzt, wird auch hier wieder die 3. Person
Einzahl des Verbs verwendet.
\begin{ejemplos}
    \item Lee ese libro, es muy bueno.
\end{ejemplos}
Wenn man mehrere Personen duzt und anspricht, wird die 
Endung der Grundform durch \textit{-ed} ersetzt.
In Lateinamerika wird hingegen für diese Form \textit{usedes}
genutzt, also wird die Endung der Grundform durch \textit{-an} ersetzt.
\begin{ejemplos}
    \item Coman el pescado.
\end{ejemplos}
Wenn man mehrere Personen anspricht und sich selbst einbezieht,
wird die Endung der Grundform durch \textit{-amos} ersetzt.
\begin{ejemplos}
    \item Vamos a llegar tarde, corramos!
\end{ejemplos}
\subsection*{Regelmäßige Verben auf -ir}
Die Endungen der regelmäßigen Verben auf \textit{-ir} sind
\textbf{gleich} wie bei den Verben auf \textit{-er}.
\begin{ejemplos}
    \item Vive saludablemente, te hace bien.
    \item Abramos las ventanas para que aire fresco.
\end{ejemplos}
Wenn man mehrere Personen duzt und anspricht, wird die 
Endung der Grundform durch \textit{-id} 
ersetzt.\footnote[1]{Das ist die einzige Endung, 
in der ein \textbf{i} vorkommt} In Lateinamerika wird 
hingegen für diese Form \textit{usedes} genutzt, 
also wird die Endung der Grundform durch \textit{-an} ersetzt.
\begin{ejemplos}
    \item Para llegar a mi casa, subid el autob\'us.
    \item Para llegar a mi casa, suban el autob\'us.
\end{ejemplos}
\subsubsection*{Vokalwechsel}
Auch ihnm Imperativ müssen bei einigen Verben der Vokalwechsel 
beachtet werden.
\begin{ejemplos}
    \item t\'u - duerme
    \item usted - duerma
    \item nosotros/as - durmamos
    \item usteded - duerman
\end{ejemplos}
\section{Verdopplung der Objektpronomen}
\textbf{03.01.2024}\\
\subsection*{Direkte Objektpronomen}
Wenn ein Satz mit einem direkten Objektpronomen beginnt,
wird danach oft ein direktes Objektpronomen wie \textit{las, lo..}
ergänzt. Das Objekt wird also verdoppelt.
\begin{ejemplos}
    \item A Pablo lo veo todos los d\'ias.
    \item Las reservaciones las haces ma\~nana.
\end{ejemplos}
Bei \textit{todo/a/s} als direktes Objekt gibt es auch eine
Verdopplung.
\begin{ejemplos}
    \item Las chaquetas? Las vendimos todas.
    \item Lo saben todo.
\end{ejemplos}
\subsection*{Indirekte Objektpronomen}
Auch die \textbf{indirekten Objektpronomen} werden häufig
verdoppelt.
\begin{ejemplos}
    \item A Manuel ya le compr\'e el billete.
    \item Ya te dieron el regalo a ti?
    \item Le estoy haciendo una torta a mi novio.
\end{ejemplos}
\section{Die Verkleinerungsform \textit{-ito, cito}}
\subsection*{Die Verkleinerungsform bei Nomen und Adjektiven}
Nomen, Adjektive wund zum Teil auch Adverbien können durch
die Endung \textit{-ito/a/s} verkleinert werden, je nach Geschlecht 
und Anzahl.
\begin{ejemplos}
    \item Espera una momentito.
    \item El h\'amster es un animalito peque\~no.
    \item Mi casa es peque\~nita.
\end{ejemplos}
Bei Wörtern auf \textit{-e, -r, -n} wird \textit{-cito}
angehängt.
\begin{ejemplos}
    \item Mi amorcito, a ti te quiero.
    \item Me llevo estos postrecitos.
\end{ejemplos}
Bei Wörtern, die auf \textit{-z} enden, wird das \textit{z} zu
einem \textit{c}.
\begin{ejemplos}
    \item el l\'apiz - el lapicito
\end{ejemplos}
Bei einigen Wörtern muss der Umlaut erhalten bleiben.
\begin{ejemplos}
    \item el amigo - el amiguito
\end{ejemplos}
\subsection*{Die Verkleinerungsform bei Adverbien}
Die Verkleinerungsform bei Adverbien bleibt unveränderlich
und endet immer auf \textit{-ito/-ita}.
\begin{ejemplos}
    \item La ni\~na corre rapidito.
    \item Tu casa est\'a cerquita de le m\'ia.
    \item Siempre nos despertamos tempranito.
\end{ejemplos}
\section{Pr\'eterito Indefinido}
Das \textit{pr\'eterito indefinido} wird verwendet, um
über Ereignisse in der Vergangenheit zu sprechen, die
\textbf{abgeschlossen} sind.
\begin{ejemplos}
    \item Ayer com\'i melocot\'on.
    \item Naci\'o en 1999.
\end{ejemplos}
\subsection*{Verben auf \textit{-ar}}
\begin{gramatica}
    \item yo → -\'e
    \item tu → -aste
    \item \'el/ella/used → -\'o
    \item nosotros/as → -amos
    \item vosotros/as → -asteis
    \item ellos/ellas/usteded → -aron
\end{gramatica}
\subsection*{Verben auf \textit{-er,ir}}
\begin{gramatica}
    \item yo → -\'i
    \item tu → -iste
    \item \'el/ella/used → -i\'o
    \item nosotros/as → -imos
    \item vosotros/as → -isteis
    \item ellos/ellas/usteded → -ieron
\end{gramatica}