\chapter{Babbel A2}
\section{Die indirekten Objektpronomen}
\textbf{31.12.20.23}\\
Nach dem \textbf{indirekten Objektpronomen} fragt man mit 
\textbf{wem oder was?}. DIe Objektpronomen für 
\textit{yo, t\'u, nosotros/as} und \textit{vosotros/as} sind 
gleich wie die direkten Objektpronomen.
\begin{ejemplos}
    \item Ella me da la flor.
    \item Andr\'es te escribe cartas?
\end{ejemplos}
\subsection*{le und les}
Die indirekten Objektpronomen für die 3. Person sind 
\underline{anders} als die direkten Objektpronomen und
richten sich allein nach der \underline{Anzahl}.
\begin{ejemplos}
    \item Le compro una manzana.
    \item Ya les hemos tra\'ido sus cosas.
    \item Nos les has comprado el billete.
\end{ejemplos}
\section{Fragewörter}
Es gibt die Fragewörter:
\begin{gramatica}
    \item was, welcher - qu\'e
    \item wie - c\'o
    \item warum - por qu\'e
    \item wann - cu\'ando
    \item d\'onde - wo
    \item wer - qui\'en
    \item wie viel - cu\'anto
\end{gramatica}
\textit{Qui\'en} und \textit{cu\'anto} müssen ans Nomen 
angeglichen werden. Für \textit{qui\'en} gibt eine
Einzahlform und eine Mehrzahlform.
\begin{ejemplos}
    \item Qui\'en eres?
    \item Qui\'enes son ustedes?
\end{ejemplos}
\textit{Cu\'anto} hingegen wird wie ein Artikel angeglichen.
Die Formen sind \textit{cu\'anto, cu\'anta, cu\'antos, cu\'antas}. 
\begin{ejemplos}
    \item Cu\'antas hermanas tienes?
    \item Cuanta gente esta aqui?
\end{ejemplos}
\section{Nomen mit unregelmäßiger Endung}
Einige Berufsbezeichnungen enden in der männlichen und 
weiblichen Form aus \textit{-a}.
\begin{ejemplos}
    \item la periodista - el periodista
    \item la artista - el artista
    \item la dentista - el dentista
\end{ejemplos}
Einige weitere Nomen auf \textit{-a} sind ebenfalls männlich.
\begin{ejemplos}
    \item el tema
    \item el problema
    \item el mapa
    \item el sofa
    \item el idioma
\end{ejemplos}
Einige Wörter, die aus \textit{-o} enden, sind dagegen 
weiblich.
\begin{ejemplos}
    \item la foto
    \item la moto
    \item la mano
\end{ejemplos}
\subsection*{Weitere Endungen}
Männliche Nomen sind durch die Endungen \textit{-or, \'on, -an}
erkennbar.
\begin{ejemplos}
    \item el coraz\'on - el sill\'on
    \item el escritor - el vendedor
    \item el capit\'an
\end{ejemplos}
Nomen mit Endungen auf \textit{-i\'on, -ad, -ez} sind fast
immer weiblich.
\begin{ejemplos}
    \item la canci\'on
    \item la ciudad
    \item la timidez - la ni\~nes
\end{ejemplos}
\section{Relativsätze mit \textit{que}}