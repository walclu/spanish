\chapter{Babbel A2}
\section{Die indirekten Objektpronomen}
\textbf{31.12.20.23}\\
Nach dem \textbf{indirekten Objektpronomen} fragt man mit 
\textbf{wem oder was?}. DIe Objektpronomen für 
\textit{yo, t\'u, nosotros/as} und \textit{vosotros/as} sind 
gleich wie die direkten Objektpronomen.
\begin{ejemplos}
    \item Ella me da la flor.
    \item Andr\'es te escribe cartas?
\end{ejemplos}
\subsection*{le und les}
Die indirekten Objektpronomen für die 3. Person sind 
\underline{anders} als die direkten Objektpronomen und
richten sich allein nach der \underline{Anzahl}.
\begin{ejemplos}
    \item Le compro una manzana.
    \item Ya les hemos tra\'ido sus cosas.
    \item Nos les has comprado el billete.
\end{ejemplos}
\section{Fragewörter}
Es gibt die Fragewörter:
\begin{gramatica}
    \item was, welcher - qu\'e
    \item wie - c\'o
    \item warum - por qu\'e
    \item wann - cu\'ando
    \item d\'onde - wo
    \item wer - qui\'en
    \item wie viel - cu\'anto
\end{gramatica}
\textit{Qui\'en} und \textit{cu\'anto} müssen ans Nomen 
angeglichen werden. Für \textit{qui\'en} gibt eine
Einzahlform und eine Mehrzahlform.
\begin{ejemplos}
    \item Qui\'en eres?
    \item Qui\'enes son ustedes?
\end{ejemplos}
\textit{Cu\'anto} hingegen wird wie ein Artikel angeglichen.
Die Formen sind \textit{cu\'anto, cu\'anta, cu\'antos, cu\'antas}. 
\begin{ejemplos}
    \item Cu\'antas hermanas tienes?
    \item Cuanta gente esta aqui?
\end{ejemplos}
\section{Nomen mit unregelmäßiger Endung}
Einige Berufsbezeichnungen enden in der männlichen und 
weiblichen Form aus \textit{-a}.
\begin{ejemplos}
    \item la periodista - el periodista
    \item la artista - el artista
    \item la dentista - el dentista
\end{ejemplos}
Einige weitere Nomen auf \textit{-a} sind ebenfalls männlich.
\begin{ejemplos}
    \item el tema
    \item el problema
    \item el mapa
    \item el sofa
    \item el idioma
\end{ejemplos}
Einige Wörter, die aus \textit{-o} enden, sind dagegen 
weiblich.
\begin{ejemplos}
    \item la foto
    \item la moto
    \item la mano
\end{ejemplos}
\subsection*{Weitere Endungen}
Männliche Nomen sind durch die Endungen \textit{-or, \'on, -an}
erkennbar.
\begin{ejemplos}
    \item el coraz\'on - el sill\'on
    \item el escritor - el vendedor
    \item el capit\'an
\end{ejemplos}
Nomen mit Endungen auf \textit{-i\'on, -ad, -ez} sind fast
immer weiblich.
\begin{ejemplos}
    \item la canci\'on
    \item la ciudad
    \item la timidez - la ni\~nes
\end{ejemplos}
\section{Relativsätze mit \textit{que}}
\textbf{01.01.2024}\\
\textit{Que} bezieht sich auf ein Nomen, das näher bestimmt
wird.
\begin{ejemplos}
    \item Tienes la revista que le\'imos ayer?
    \item El amigo que me llam\'o por la ma\~nana viene
    ahora.
    \item Los ni\~nos que juegan ah\'i son mis primos.
    \item La maleta que voy a comprar es peque\~na.
\end{ejemplos}
\section{Desde und desde hace}
Die deutsche Präposition \textbf{seit} hat im Spanischen
zwei Bedeutungen. \textit{Desde hace} bezieht sich auf einen
Zeitraum, der in der Vergangenheit angefangen hat.
\begin{ejemplos}
    \item Desde hace dos semanas estoy mejor.
    \item No voy al m\'edico desde hace cuatro mes.
\end{ejemplos}
Ein bestimmter \underline{Zeitpunkt} in der Vergangenheit
wird dagegen mit \textit{desde} angegeben.
\begin{ejemplos}
    \item No fumo desde el a\~no pasado.
    \item Desde el sabado estoy enfermo.
\end{ejemplos}
\subsection*{Zeitangaben mit \textit{hace}}
\textit{Hace} wird wie das Deutsche \textbf{vor} verwendet,
um abgeschlossene Zeiträume in der Vergangenheit zu beschreiben.
\begin{ejemplos}
    \item Hace dos semanas fui al gimnasio.
    \item Comimos hace tres horas.
    \item Hace unos meses estuvo enfermo.
\end{ejemplos}
\section{Imperativ}
\subsection*{Regelmäßige Verben auf -ar}
Die Befehlsform wird verwendet, um jemanden um etwas direkt
zu bitten oder ihn aufzufordern. Wenn man eine Person duzt,
benutzt man einfach die dritte Person Einzahl des Verbes.
\begin{ejemplos}
    \item Jordi no canta bien, canta t\'u.
\end{ejemplos}
Wenn man mehrere Personen duzt und anspricht, wird die 
Endung der Grundform durch \textit{-ad} ersetzt.
\begin{ejemplos}
    \item No pod\'eis jugar aqui, jugad en el parque.
\end{ejemplos}
In Lateinamerika wird hingegen für diese Form 
\textit{usedes} genutzt, also wird die Endung 
der Grundform durch \textit{-en} ersetzt.
\begin{ejemplos}
    \item No pueden cantar aqui, canten en el parque.
\end{ejemplos}
Wenn man mehrere Personen anspricht und sich selbst einbezieht,
wird die Endung der Grundform durch \textit{-emos} ersetzt.
\begin{ejemplos}
    \item Ellos no saben c\'omo limpiar, limpiemos la casa.
\end{ejemplos}
\subsection*{Regelmäßige Verben auf -er}
Wenn man eine Person duzt, wird auch hier wieder die 3. Person
Einzahl des Verbs verwendet.
\begin{ejemplos}
    \item Lee ese libro, es muy bueno.
\end{ejemplos}
Wenn man mehrere Personen duzt und anspricht, wird die 
Endung der Grundform durch \textit{-ed} ersetzt.
In Lateinamerika wird hingegen für diese Form \textit{usedes}
genutzt, also wird die Endung der Grundform durch \textit{-an} ersetzt.
\begin{ejemplos}
    \item Coman el pescado.
\end{ejemplos}
Wenn man mehrere Personen anspricht und sich selbst einbezieht,
wird die Endung der Grundform durch \textit{-amos} ersetzt.
\begin{ejemplos}
    \item Vamos a llegar tarde, corramos!
\end{ejemplos}
\subsection*{Regelmäßige Verben auf -ir}
Die Endungen der regelmäßigen Verben auf \textit{-ir} sind
\textbf{gleich} wie bei den Verben auf \textit{-er}.
\begin{ejemplos}
    \item Vive saludablemente, te hace bien.
    \item Abramos las ventanas para que aire fresco.
\end{ejemplos}
Wenn man mehrere Personen duzt und anspricht, wird die 
Endung der Grundform durch \textit{-id} 
ersetzt.\footnote[1]{Das ist die einzige Endung, 
in der ein \textbf{i} vorkommt} In Lateinamerika wird 
hingegen für diese Form \textit{usedes} genutzt, 
also wird die Endung der Grundform durch \textit{-an} ersetzt.
\begin{ejemplos}
    \item Para llegar a mi casa, subid el autob\'us.
    \item Para llegar a mi casa, suban el autob\'us.
\end{ejemplos}
\subsubsection*{Vokalwechsel}
Auch ihnm Imperativ müssen bei einigen Verben der Vokalwechsel 
beachtet werden.
\begin{ejemplos}
    \item t\'u - duerme
    \item usted - duerma
    \item nosotros/as - durmamos
    \item usteded - duerman
\end{ejemplos}
\section{Verdopplung der Objektpronomen}
\textbf{03.01.2024}\\
\subsection*{Direkte Objektpronomen}
Wenn ein Satz mit einem direkten Objektpronomen beginnt,
wird danach oft ein direktes Objektpronomen wie \textit{las, lo..}
ergänzt. Das Objekt wird also verdoppelt.
\begin{ejemplos}
    \item A Pablo lo veo todos los d\'ias.
    \item Las reservaciones las haces ma\~nana.
\end{ejemplos}
Bei \textit{todo/a/s} als direktes Objekt gibt es auch eine
Verdopplung.
\begin{ejemplos}
    \item Las chaquetas? Las vendimos todas.
    \item Lo saben todo.
\end{ejemplos}
\subsection*{Indirekte Objektpronomen}
Auch die \textbf{indirekten Objektpronomen} werden häufig
verdoppelt.
\begin{ejemplos}
    \item A Manuel ya le compr\'e el billete.
    \item Ya te dieron el regalo a ti?
    \item Le estoy haciendo una torta a mi novio.
\end{ejemplos}
\section{Die Verkleinerungsform \textit{-ito, cito}}
\subsection*{Die Verkleinerungsform bei Nomen und Adjektiven}
Nomen, Adjektive wund zum Teil auch Adverbien können durch
die Endung \textit{-ito/a/s} verkleinert werden, je nach Geschlecht 
und Anzahl.
\begin{ejemplos}
    \item Espera una momentito.
    \item El h\'amster es un animalito peque\~no.
    \item Mi casa es peque\~nita.
\end{ejemplos}
Bei Wörtern auf \textit{-e, -r, -n} wird \textit{-cito}
angehängt.
\begin{ejemplos}
    \item Mi amorcito, a ti te quiero.
    \item Me llevo estos postrecitos.
\end{ejemplos}
Bei Wörtern, die auf \textit{-z} enden, wird das \textit{z} zu
einem \textit{c}.
\begin{ejemplos}
    \item el l\'apiz - el lapicito
\end{ejemplos}
Bei einigen Wörtern muss der Umlaut erhalten bleiben.
\begin{ejemplos}
    \item el amigo - el amiguito
\end{ejemplos}
\subsection*{Die Verkleinerungsform bei Adverbien}
Die Verkleinerungsform bei Adverbien bleibt unveränderlich
und endet immer auf \textit{-ito/-ita}.
\begin{ejemplos}
    \item La ni\~na corre rapidito.
    \item Tu casa est\'a cerquita de le m\'ia.
    \item Siempre nos despertamos tempranito.
\end{ejemplos}
\section{Pr\'eterito Indefinido}
Das \textit{pr\'eterito indefinido} wird verwendet, um
über Ereignisse in der Vergangenheit zu sprechen, die
\textbf{abgeschlossen} sind.
\begin{ejemplos}
    \item Ayer com\'i melocot\'on.
    \item Naci\'o en 1999.
\end{ejemplos}
\subsection*{Verben auf \textit{-ar}}
\begin{gramatica}
    \item yo → -\'e
    \item tu → -aste
    \item \'el/ella/used → -\'o
    \item nosotros/as → -amos
    \item vosotros/as → -asteis
    \item ellos/ellas/ustedes → -aron
\end{gramatica}
\subsection*{Verben auf \textit{-er,ir}}
\begin{gramatica}
    \item yo → -\'i
    \item tu → -iste
    \item \'el/ella/used → -i\'o
    \item nosotros/as → -imos
    \item vosotros/as → -isteis
    \item ellos/ellas/ustedes → -ieron
\end{gramatica}
\section{Indefinido vs. pret\'erito perfecto}
\textbf{05.01.2024}\\
Die spanischen Vergangenheitszeiten werden anders als das
deutsche Präteritum und Perfekt. Das \textit{indefinido}
bezieht sich auf abgeschlossene Handlungen in der Vergangenheit.
\begin{ejemplos}
    \item Ayer compr\'e un pantal\'on.
\end{ejemplos}
Das \textit{pret\'erito perfecto} beschreibt Handlungen,
die in der Vergangenheit angefangen haben und 
\underline{noch andauern} - oder zumindest stark auf die
Gegenwart auswirken.
\begin{ejemplos}
    \item He vivido por ac\'a desde hace dos a\~nos.
    \item No la has llamdo a\'un.
    \item Ya han comido algo?
\end{ejemplos}
Es gibt zusätzlich noch Zeitmarker, die einem die eindeutige
Verwendung erleichern.
\begin{ejemplos}
    \item \textbf{La semana pasada} gan\'e tres veces al ajedrez.
    \item \textbf{Esta semana} he ganado tres veces al ajedrez.
\end{ejemplos}
Alle Zeitmarker, die \textit{pasado}, \textit{ayer} und ähnliche Bezüge beinhalten
oder sich auf einen bestimmten Zeitpunkt in der Vergangenheit
beziehen, verlangen immer das \textit{indefinido}.
\begin{ejemplos}
    \item Anoche jugaron a las cartas.
\end{ejemplos}
Alle Zeitmarker, die \textit{este/esta}, \textit{hoy}, \textit{ya}
beinhalten oder sich auf \textbf{bisher} beziehen, verlangen
den \textit{pret\'erito perfecto}.
\begin{ejemplos}
    \item A\'un no he entendido las regalas del juego.
    \item Todav\'ia no he comprado el regalo.
\end{ejemplos}
Manchmal kann man \textit{ya} mit \textit{indefinido} verwenden,
wenn die Abgeschlossenheit des Ereignissen betont werden soll.
\begin{ejemplos}
    \item Ya lo hice ayer.
\end{ejemplos} 
\section{Das \textit{futuro simple}}
Das \textit{futuro simple} wird durch das Anhängen bestimmter 
Endungen an die \underline{Grundform} gebildet.
\begin{ejemplos}
    \item No trabajaremos en diciembre.
    \item Ma\~nana dormir\'e hasta las 11.
    \item Ellos me llamar\'an pronto.
\end{ejemplos}
\subsection*{Bildung der Formen}
\begin{gramatica}
    \item yo → -\'e
    \item tu → -as
    \item \'el/ella/used → -\'a
    \item nosotros/as → -emos
    \item vosotros/as → -\'eis
    \item ellos/ellas/ustedes → -\'an
\end{gramatica}
\subsection*{Unregelmäßige Zeitformen}
Bei einigen Verben wird die Grundform leicht verändert,
um die Zukunft zu bilden. \textit{Hacer} und \textit{Decir}
verwandeln z.B. ihre Grundform in der Zukunft in \textit{har-}
und \textit{dir-}.
\begin{ejemplos}
    \item El viernes har\'a sol.
    \item Nosotros no diremos nada.
\end{ejemplos}
\textit{tener}, \textit{salir} und \textit{venir} ersetzen
die Endung \textit{er/ir} durch \textbf{dr}.
\begin{ejemplos}
    \item Ella pr\'oximamente tendra ex\'amenes.
    \item Nosotros saldremos el viernes.
    \item Usedes vendr\'an pronto.
\end{ejemplos}
\textit{Saber} und \textit{poder} verlieren das \textbf{e}.
\begin{ejemplos}
    \item T\'u  sabras mucho.
    \item Ella no podr\'an venir.
\end{ejemplos}
Es macht keinen Unterschied welche zwei Zukunftsformen man 
verwendet.
\section{Vergleiche mit dem Komparativ}
\textbf{06.01.2024}\\
Um den Komparativ zu bilden benutzt man folgende Form.
\begin{gramatica}
    \item m\'as + Adj + que
    \item menos + Adj + que
\end{gramatica}
\textit{m\'as} oder \textit{menos} stehen immer \textbf{vor}
dem Adjektiv und \textit{que} danach.
\begin{ejemplos}
    \item T\'u eres m\'as alto que yo.
    \item T\'u eres menos listo que yo.
\end{ejemplos}
Man kann so auch Vergleiche mit \textbf{Nomen} und Verben
anstellen.
\begin{ejemplos}
    \item Yo tengo m\'as libros que t\'u.
    \item Ella lee m\'as que yo.
\end{ejemplos}
\subsection*{Vergleich gleichwertiger Dinge}
Wenn man zwei gleichwertige Dinge miteinander vergleichen
möchte, verwendet man folgende Form.
\begin{gramatica}
    \item tan + Adj + como
\end{gramatica}
\begin{ejemplos}
    \item T\'u eres tan listo como \'el.
    \item Yo soy tan guapo como t\'u.
\end{ejemplos}
Bei Verben wird diese Form angepasst.
\begin{gramatica}
    \item tanto + como
\end{gramatica}
\begin{ejemplos}
    \item Yo leo tanto como t\'u.
\end{ejemplos}
Bei Nomen verhält sich \textit{tanto} wie ein Artikel und muss
angeglichen werden. Die Formen sind daher 
\textit{tanto, tanta, tantos, tantas}.
\begin{ejemplos}
    \item Yo tengo tanto dinero como t\'u.
    \item Yo tengo tantas amigas como t\'u.
\end{ejemplos}
\section{Vergleiche mit dem Superlativ}
Um die höchste Steigerungsform, den \textbf{Superlativ}, zu 
bilden, braucht man ebenfalls \textit{m\'as} oder \textit{menos}
sowie die Artikel \textit{el, ella, los, las}.
\begin{ejemplos}
    \item Yo soy el m\'as listo.
    \item T\'u eres la menos ego\'ista. 
\end{ejemplos}
\subsection*{Unregelmäßige Formen}
\textit{bueno} und \textit{malo} haben \underline{Unregelmäßige}
Komparativ- und Superlativformen.
\begin{ejemplos}
    \item Este libro es mejor que aquel.
    \item Esta pel\'icula es peor que la de ayer.
\end{ejemplos}
\textit{bueno} und \textit{malo} haben nur eine Form für
den Komparativ und Superlativ. Diese sind \textit{mejor} bzw.
\textit{peor}.
\begin{ejemplos}
    \item Este libro es el mejor.
    \item Esta pel\'icula es la peor.
\end{ejemplos}
\section{Adverbien}
Alle Adjektive werden an das Nomen oder Pronomen angeglichen,
das sie bestimmt.
\begin{ejemplos}
    \item Este auto es muy r\'apido.
    \item La chaqueta nueva est\'a en el armario.
\end{ejemplos}
Adverbien sind \textbf{unveränderlich}. Sie beziehen sich
auf Verben, Adjektive, andere Adverbien oder ganze Sätze.
\begin{ejemplos}
    \item Yo leo r\'apidamente.
    \item Solo vamos a comer paella.
\end{ejemplos}
\subsection*{Bildung der Adverbien}
Um aus Adjektiven, die auf \textit{-e} oder Konsonant enden,
ein Adverb zu bilden, wird \textit{-mente} angehängt.
\begin{ejemplos}
    \item Frecuentemente voy a nadar.
    \item Me duermo f\'acilmente.
\end{ejemplos}
An Adjektive, die auf \textit{-o/-a} enden, wird \textit{-mente}
and die \textbf{weibliche Form} angehängt.
\begin{ejemplos}
    \item \'El va a volver seguramente.
    \item Me fui r\'apidamente.
\end{ejemplos}
\subsection*{Unregelmäßige Formen}
Die Verben \textit{r\'apido} und \textit{lento} können auch als Adverb
genutzt werden.
\begin{ejemplos}
    \item Nadas r\'apido.
    \item Corres lento.
\end{ejemplos}
Auch \textit{bueno} und \textit{malo} bilden das Adverb nicht
regelmäßig. Die Adverbien lauten \textit{bien} und \textit{mal}.
\begin{ejemplos}
    \item Nosotros estamos bien.
    \item Vosotros no jug\'ais mal.
\end{ejemplos}
\section{muy und mucho}
\textit{muy} und \textit{mucho} können beide \underline{sehr} 
heißen. Vor Adverbien und Adjektiven steht immer \textit{muy}.
\begin{ejemplos}
    \item Es muy tarde.
    \item Nosotros leemos muy r\'apido.
\end{ejemplos}
Bei Verben steht \textit{mucho} unveränderlich im Satz.
\begin{ejemplos}
    \item Lo siento mucho.
    \item Yo os quiero mucho.
\end{ejemplos}
Bei Nomen wird \textit{mucho} angeglichen.
\section{por und para}
\textbf{07.01.2023}\\
Um \textbf{Zweck}, \textbf{Empfänger} oder \textbf{Zielort}
auszudrücken, verwendet man die Präposition \textit{para}.
\begin{ejemplos}
    \item Necesito gafas para ver mejor.
    \item Esto es para tu hermana.
    \item Vamos para Madrid.
\end{ejemplos}
\textit{por} wird hingegen verwendet, um \textbf{Gründe},
\textbf{Preise} und \textbf{Tageszeiten} zu beschreiben, aber auch
\textbf{womit} oder \textbf{wodurch} man etwas macht.
\begin{ejemplos}
    \item Por tu demora vamos a llegar tarde.
    \item Lo compraron por poco dinero.
    \item No ves televisi\'on la ma\~nana.
    \item Te llam\'e por tel\'efono.
\end{ejemplos}
\section{Befehlsformen und Pronomen}
Objektpronomen und Reflexivpronomen hängt man an die Befehlsform,
wenn sie \underline{nicht} verneint ist.
\begin{ejemplos}
    \item Dame un consejo, por favor.
    \item L\'avate las manos antes de comer.
\end{ejemplos}
Treten indirekte und direkte Objektpronomen zusammen auf,
steht die in direkte Form vor der direkten. Die Reihenfolge
ist also \underline{genau umgekehrt} wie im Deutschen.
\begin{ejemplos}
    \item Ah\'i est\'an mis gafas. D\'amelas, por favor.
    \item Coge tu sombrero y p\'ontelo.
\end{ejemplos}
\subsection*{Weitere Regeln zur Befehlsform und Pronomen}
Hängt man das Pronomen \textit{os} an die Befehlsform mit den
Endungen \textit{-ad,-ed,-id}, so wird as \textit{-d} 
dieser Endung gestrichen.
\begin{ejemplos}
    \item Vest\'ios r\'apido, tenemos que salir ya.
\end{ejemplos}
Wird das Pronomen \textit{nos} an die Befehlsform mit den
Endungen \textit{-amos,-emos} gehängt, so fällt das \textbf{s}
der Endung weg.
\begin{ejemplos}
    \item Sent\'emonos aqu\'i.
\end{ejemplos}
\section{Unregelmäßige Befehlsformen}
Oft ist die Befehlsform, um eine einzelne Person anzusprechen,
die man duzt, \textbf{unregelmäßig}.
\begin{ejemplos}
    \item Te invito a cenar, ven a mi casa.
\end{ejemplos}
Bei allen unregelmäßigen Verben sind die Formen der Befehlsform
zum Ansprechen einer Gruppe, in der man sich nicht einbezieht,
\textbf{regelmäßig}.
\begin{ejemplos}
    \item Hoy hay una fiesta, vengan todos.
    \item Yo no voy a la fiesta, vayan ustedes.
\end{ejemplos}
\section{\textit{Condicional}}
Im Spanischen verwendet man das \textit{condicional}, um 
\textbf{höflich} um etwas zu bitten.
\begin{ejemplos}
    \item Ser\'ia tan amable de ayudarme?
\end{ejemplos}
Desweiteren gebraucht man das Konditional, um 
\textbf{einen Wunsch} zu äußern.
\begin{ejemplos}
    \item Me gustar\'ia ir al teatro esta noche.
\end{ejemplos}
Das \textit{condicional} wird auch verwendet, um 
eine \textbf{Vermutung} auszudrücken.
\begin{ejemplos}
    \item Creo que te ser\'ia \'util aprender espa\~nol.
\end{ejemplos}
Zuletzt ist es auch üblich, mithilfe des \textit{condicional}
Ratschläge zu geben.
\begin{ejemplos}
    \item Deber\'ias ir al m\'edico.
\end{ejemplos}
\subsection*{Regelmäßige Formen des \textit{condicional}}
\begin{gramatica}
    \item yo - \'ia
    \item tu - \'ias
    \item \'el/ella/usted - \'ia
    \item nosotras/as - \'iamos
    \item vosotros/as - \'iais
    \item ellos/ellas/ustedes - \'ian
\end{gramatica}
\subsection*{Unregelmäßige Formen des \textit{condicional}}
Bei einigen Verben werden die \textit{condicional}-Endungen 
an eine veränderte Grundform gehängt. Diese entspricht den
veränderten Grundformen bei unregelmäßigen Verben des 
\textit{futuro simple}.
\begin{ejemplos}
    \item Podr\'ias darme eso.
    \item Ellos no har\'ian una obra para ni\~nos.
    \item No querr\'iamos trabajar por la noche.
    \item Qu\'e dir\'ia el director?
\end{ejemplos}
Bei anderen Verben fällt das e in der Grundform auch weg,
wird aber \textbf{d} ersetzt.
\begin{ejemplos}
    \item Tendr\'ias tiempo en el intermedio?
\end{ejemplos}
\section{Plusquamperfekt}
\textbf{09.01.2024}\\
Um das Plusquamperfekt zu bilden, braucht man immer das Hilfsverb
\textit{haber} im Imperfekt und das Partizip.
\begin{gramatica}
    \item haber (imperfecto) + participio
\end{gramatica}
\begin{ejemplos}
    \item Luis hab\'ia lavado toda la vajilla.
\end{ejemplos}
Das Plusquamperfekt drückt eine Handlung aus, die \underline{schon}
abgeschlossen ist und noch vor einer anderen Handlung in der
Vergangenheit stattgefunden hat. Egal in welcher Vergangenheitsform
der andere Teilsatz steht, das Plusquamperfekt drückt immer eine
Vorvergangenheit aus.
\section{Das Wort \textit{deber}}
In bejahended Aussagen gibt man mit \textit{deber} an, dass 
etwas getan werden \underline{soll oder muss}.
\begin{ejemplos}
    \item Lidia dice que debo que limpiar la casa.
\end{ejemplos}
Wird \textit{deber} verneint, dann bedeutet es meistens 
\underline{nicht sollen oder nicht dürfen}.
\begin{ejemplos}
    \item No debo ver la televisi\'on.
\end{ejemplos}
Wird \textit{deber} allein ohne andere Verben in der Grundform
verwendet, so kann es auch \underline{jmd. etwas schulden} oder
\underline{jmd. für etwas danken} heißen.
\begin{ejemplos}
    \item Le debo 500€.
    \item Le debo mucho a mi madre. 
\end{ejemplos}

\section{Die Endung -\textit{\'ismo} bei Adjektiven}
\textbf{12.01.2024}\\
Oft hängt man in der Umgangssprache an Adjektive
die Endung \textit{-\'ismo}, um den 
\underline{höchsten} Grad einer Eigenschaft auszudrücken.
\begin{ejemplos}
    \item Felipe est\'a enamorad\'isimo.
\end{ejemplos}
Bei Adjektiven, die auf Vokale \textbf{a,e,o} enden,
streicht man die Endung weg und hängt \textit{-\'isimo}
an. Bei Adjektiven, die nicht auf einen Vokal enden,
wird \textit{\'isimo} direkt angehangen.
\begin{ejemplos}
    \item Eso fue facil\'isimo.
\end{ejemplos}
\subsection*{Die Angleichung}
Adjektive auf \textit{-\'isimo} werden in Geschlecht 
und Zahl an das Nomen angeglichen.
\begin{ejemplos}
    \item Esa fue una relaci\'on dificil\'isma.
    \item Tus citas son interesant\'isimas.
\end{ejemplos}
\section{\textit{traer} und \textit{llevar}}
\subsection*{Die Verwendung von \textit{traer}}
Das Verb \textit{traer} kann mit \textbf{bringen}
oder \textbf{mitbringen} übersetzt werden. Im Gegensatz
zum Deutschen hängt die Verwendung vom Sprecher ab,
die Bewegung geht zu diesem hin, im Sinne von 
\textbf{her}bringen.
\begin{ejemplos}
    \item Puedes traer un postre?
    \item Nos van a traer la comida a casa.
\end{ejemplos}
\subsection*{Die Verwendung von \textit{llevar}}
Das Verb \textit{llevar} bedeutet \textbf{bringen zu},
\textbf{mitbringen} in Bezug auf \textbf{hinbringen} sowie
mitnehmen.
\begin{ejemplos}
    \item Voy a llevar una ensalada a tu casa.
    \item Ll\'evate la comida.
\end{ejemplos}